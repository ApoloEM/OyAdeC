%==================================================================
% CLASE DEL DOCUMENTO
%==================================================================
\documentclass[a4paper,12pt]{article}

%==================================================================
% IDIOMA Y CODIFICACIÓN
%==================================================================
\usepackage[utf8]{inputenc}
\usepackage[T1]{fontenc}
\usepackage[spanish, es-tabla]{babel}

%==================================================================
% FORMATO GENERAL Y HERRAMIENTAS
%==================================================================
\usepackage{geometry}
\geometry{left=2.5cm, right=2.5cm, top=3cm, bottom=3cm}

\usepackage{graphicx} 
\usepackage{float}    
\usepackage{enumitem} 
\usepackage{listings} 
\usepackage{xcolor}   
\usepackage{tikz}     
\usepackage{eso-pic}  
\usepackage{titlesec} 
\usepackage{hyperref}
\usepackage{lipsum} % paquete para texto de relleno
\usetikzlibrary{calc}

%==================================================================
% COLORES Y ESTILO VISUAL (Personalización Estética)
%==================================================================
\definecolor{VerdeUABC}{RGB}{75, 83, 32} % Tu color VerdeMilitar
\definecolor{backcolour}{rgb}{0.96,0.96,0.94}
\definecolor{codegreen}{rgb}{0,0.6,0}

% Estilo de código optimizado para Ensamblador y C
\lstset{
    backgroundcolor=\color{backcolour},   
    commentstyle=\color{codegreen},
    keywordstyle=\color{magenta},
    numberstyle=\tiny\color{gray},
    stringstyle=\color{purple},
    basicstyle=\ttfamily\small,
    breaklines=true,
    frame=single,
    captionpos=b,
    language=[x86masm]Assembler % Lenguaje por defecto
}

\hypersetup{
    colorlinks=true,
    linkcolor=black,
    urlcolor=blue,
    citecolor=black
}

%==================================================================
% ENCABEZADO Y PIE DE PÁGINA
% Configuración con fancyhdr
% NO MODIFICAR si no se conoce fancyhdr
%==================================================================
\usepackage{fancyhdr}
\pagestyle{fancy}
\fancyhf{}
\lhead{\footnotesize OyAdeC}
\chead{\footnotesize \tituloPractica}
\rhead{\footnotesize 1110604 - EAMM}
\cfoot{\thepage}

% Configuración de títulos
\titleformat{\section}{\large\bfseries}{\thesection.}{1em}{}
\titleformat{\subsection}{\normalsize\bfseries}{\thesubsection.}{1em}{}

%==================================================================
% DATOS DEL REPORTE (MODIFICAR AQUÍ)
%==================================================================
\newcommand{\tituloPractica}{Título de la Práctica}
\newcommand{\numeroPractica}{\#XX}
\newcommand{\unidadNumero}{X}
\newcommand{\profesor}{Ing. Felix Arredondo Santamaria}
\newcommand{\fechaEntrega}{DD/FEBRERO/2026}

% --- INTEGRANTES ---
\newcommand{\integranteA}{Moya Monreal Erick Anselmo --- 1110604}

%==================================================================
% MARCO DECORATIVO
%==================================================================
\newcommand{\MarcoPagina}{
    \begin{tikzpicture}[remember picture, overlay]
        \draw [line width=1.2pt, color=VerdeUABC!90] 
            ($(current page.north west) + (0.7cm,-0.7cm)$) rectangle ($(current page.south east) + (-0.7cm,0.7cm)$);
        \draw [line width=0.6pt, color=VerdeUABC!60] 
            ($(current page.north west) + (0.85cm,-0.85cm)$) rectangle ($(current page.south east) + (-0.85cm,0.85cm)$);
    \end{tikzpicture}
}

%==================================================================
% INICIO DEL DOCUMENTO
%==================================================================
\begin{document}

% 1. PORTADA
\begin{titlepage}
    \AddToShipoutPictureBG*{\MarcoPagina}
	\begin{center}
        \includegraphics[width=4cm]{C:/Users/erick/Downloads/organizacion_y_Arquitectura_de_computadoras/laboratorio/recursos/escudo-actualizado-2022.png}
		\hspace{1cm}
		\includegraphics[width=5cm]{C:/Users/erick/Downloads/organizacion_y_Arquitectura_de_computadoras/laboratorio/recursos/1593484343817.jpg}
		\vspace{1.5cm}

		\LARGE \textbf{Universidad Autónoma de Baja California}\\
		\Large Facultad de Ingeniería Mexicali\\
		\Large Ingeniería en Computación\\
		\vspace{0.7cm}

		{\color{VerdeUABC}\rule{\linewidth}{0.5mm}} \\
		\large \textbf{Reporte de Práctica: \numeroPractica}\\
		\vspace{0.1cm}
		{\Huge {\tituloPractica} }\\
		{\color{VerdeUABC}\rule{\linewidth}{0.5mm}}
        \vfill
		Organización y Arquitectura de Computadoras\\
        \profesor \\
        \vspace{2cm}
        \integranteA \\
        \vspace{1.4cm}
		\textbf{Fecha de entrega:} \\ \fechaEntrega \\
        \vspace{1cm}
        {\color{VerdeUABC}\rule{4cm}{0.1mm}}\\
		Mexicali, Baja California\\ \today
	\end{center}
\end{titlepage}

% ÍNDICE
\newpage
\AddToShipoutPictureBG{\MarcoPagina}
\tableofcontents

% 2. INTRODUCCIÓN
\newpage
\section{Introducción}
Escribe aquí el contexto general de la práctica y un resumen de lo que se abordará.

% 3. PROBLEMÁTICA
\section{Problemática}
Describe el reto técnico o el problema específico que se busca resolver en esta sesión de laboratorio.

% 4. JUSTIFICACIÓN
\section{Justificación}
Explica por qué es importante resolver esta problemática y cómo ayuda a entender la arquitectura de computadoras.

% 5. OBJETIVOS DE LA PRÁCTICA
\section{Objetivos de la práctica}
\begin{itemize}
    \item \textbf{General:} Diseñar e implementar...
    \item \textbf{Específicos:}
    \begin{itemize}
        \item Manipular registros del procesador...
        \item Implementar lógica en ensamblador inline...
    \end{itemize}
\end{itemize}

% 6. MARCO TEÓRICO
\section{Marco teórico}
Conceptos clave como: Pila de registros, FPU, direccionamiento indirecto, etc.
\cite{ref_irvine} % Ejemplo de cita

% 7. PROCEDIMIENTO
\section{Procedimiento}
Describe paso a paso lo realizado. Puedes incluir diagramas de flujo o bloques.


[Image of a computer architecture flowchart]


\subsection{Código Destacado}
\begin{lstlisting}[language=C, caption=Integración de ensamblador en C]
__asm {
    mov eax, 1
    add eax, ebx
}
\end{lstlisting}

% 8. RESULTADOS CON PRUEBAS
\section{Resultados con pruebas}
Inserta aquí capturas de pantalla de la ejecución, tablas de datos o fotos del hardware si aplica.

\begin{figure}[H]
    \centering
    %\includegraphics[width=0.8\textwidth]{resultado1.png}
    \caption{Captura de pantalla de la ejecución exitosa.}
\end{figure}

% 9. CONCLUSIONES
\section{Conclusiones}
Reflexiones finales sobre el aprendizaje obtenido y el cumplimiento de los objetivos.

% REFERENCIAS
\newpage
\begin{thebibliography}{9}
    \bibitem{ref_irvine} 
    Irvine, K. (2019). \textit{Assembly Language for x86 Processors}. Pearson.
\end{thebibliography}

\end{document}