%==================================================================
% CLASE DEL DOCUMENTO
%==================================================================
\documentclass[a4paper,12pt]{article}

%==================================================================
% IDIOMA Y CODIFICACIÓN
%==================================================================
\usepackage[utf8]{inputenc}
\usepackage[T1]{fontenc}
\usepackage[spanish, es-tabla]{babel}

%==================================================================
% FORMATO GENERAL Y HERRAMIENTAS
%==================================================================
\usepackage{geometry}
\geometry{left=2.5cm, right=2.5cm, top=3cm, bottom=3cm}
\setlength{\parskip}{1\baselineskip}

\usepackage{microtype}
\usepackage{graphicx} 
\usepackage{float}    
\usepackage{enumitem} 
\usepackage{listings} 
\usepackage{xcolor}   
\usepackage{tikz}     
\usepackage{eso-pic}  
\usepackage{titlesec}
\usepackage{multicol}
\usepackage{hyperref}
\usepackage{lipsum}
\usepackage{booktabs}  % Para tablas profesionales
\usepackage{array}     % Para personalizar columnas de tablas
\usetikzlibrary{calc}

%==================================================================
% COLORES Y ESTILO VISUAL
%==================================================================
\definecolor{VerdeUABC}{RGB}{75, 83, 32}
\definecolor{backcolour}{rgb}{0.96,0.96,0.94}
\definecolor{codegreen}{rgb}{0,0.6,0}
\definecolor{codepurple}{rgb}{0.58,0,0.82}
\definecolor{codeorange}{rgb}{0.8,0.4,0}
\definecolor{hexcolor}{RGB}{0, 102, 204}      % Azul para hexadecimal
\definecolor{bincolor}{RGB}{153, 0, 153}      % Morado para binario

%==================================================================
% CONFIGURACIÓN DE LISTINGS PARA MÚLTIPLES LENGUAJES
%==================================================================

% Estilo base común para todos los lenguajes
\lstdefinestyle{baseStyle}{
    backgroundcolor=\color{backcolour},
    commentstyle=\color{codegreen}\itshape,
    numberstyle=\tiny\color{gray},
    basicstyle=\ttfamily\small,
    breaklines=true,
    frame=single,
    captionpos=b,
    numbers=left,
    numbersep=8pt,
    tabsize=4,
    showstringspaces=false
}

% Estilo específico para Assembly x86
\lstdefinestyle{assemblyStyle}{
    style=baseStyle,
    language=[x86masm]Assembler,
    keywordstyle=\color{blue}\bfseries,
    stringstyle=\color{codepurple},
    morekeywords={section, global, extern, db, dw, dd, dq, times, equ},
    morecomment=[l]{;},
    commentstyle=\color{codegreen}\itshape
}

% Estilo específico para C
\lstdefinestyle{cStyle}{
    style=baseStyle,
    language=C,
    keywordstyle=\color{magenta}\bfseries,
    stringstyle=\color{codepurple},
    directivestyle=\color{codeorange},
    emph={__asm, inline, register, volatile},
    emphstyle=\color{blue}\bfseries
}

% Estilo específico para C++ (por si acaso)
\lstdefinestyle{cppStyle}{
    style=baseStyle,
    language=C++,
    keywordstyle=\color{magenta}\bfseries,
    stringstyle=\color{codepurple},
    directivestyle=\color{codeorange},
    emph={__asm, inline, register, volatile, constexpr},
    emphstyle=\color{blue}\bfseries
}

% Configuración por defecto (Assembly)
\lstset{style=assemblyStyle}

%==================================================================
% COMANDOS PERSONALIZADOS PARA ARQUITECTURA
%==================================================================

% Comando para instrucciones en línea (ej: \inst{MOV}, \inst{ADD})
\newcommand{\inst}[1]{\texttt{\textbf{#1}}}

% Comando para registros en línea (ej: \reg{EAX}, \reg{EBX})
\newcommand{\reg}[1]{\texttt{\textcolor{blue}{#1}}}

% Comando para valores hexadecimales (ej: \hex{0x1A2B})
\newcommand{\hex}[1]{\texttt{\textcolor{hexcolor}{#1}}}

% Comando para valores binarios (ej: \bin{10110101})
\newcommand{\bin}[1]{\texttt{\textcolor{bincolor}{#1}}}

% Comando para formato de instrucción completo
% Uso: \instrformat{MOV}{EAX, EBX}
\newcommand{\instrformat}[2]{\texttt{\textbf{#1} #2}}

%==================================================================
% ENTORNOS PARA TABLAS DE REGISTROS
%==================================================================

% Definir anchos de columna para tablas de registros
\newcolumntype{R}{>{\centering\arraybackslash}p{2.5cm}}
\newcolumntype{V}{>{\centering\arraybackslash}p{3cm}}

\hypersetup{
    colorlinks=true,
    linkcolor=black,
    urlcolor=blue,
    citecolor=black
}

%==================================================================
% ENCABEZADO Y PIE DE PÁGINA
%==================================================================
\usepackage{fancyhdr}
\pagestyle{fancy}
\fancyhf{}
\lhead{\footnotesize OyAdeC}
\chead{\footnotesize \tituloPractica}
\rhead{\footnotesize 1110604}
\cfoot{\thepage}

% Configuración de títulos
\titleformat{\section}{\large\bfseries\color{VerdeUABC}}{\thesection.}{1em}{}
\titleformat{\subsection}{\normalsize\bfseries}{\thesubsection.}{1em}{}
\titleformat{\subsubsection}{\small\bfseries}{\thesubsubsection.}{1em}{}

%==================================================================
% DATOS DEL REPORTE (MODIFICAR AQUÍ)
%==================================================================
\newcommand{\tituloPractica}{Título de la Práctica}
\newcommand{\numeroPractica}{\#XX}
\newcommand{\unidadNumero}{X}
\newcommand{\profesor}{Ing. Felix Arredondo Santamaria}
\newcommand{\fechaEntrega}{DD/FEBRERO/2026}

\newcommand{\integranteA}{Moya Monreal Erick Anselmo --- 1110604}

%==================================================================
% MARCO DECORATIVO
%==================================================================
\newcommand{\MarcoPagina}{
    \begin{tikzpicture}[remember picture, overlay]
        \draw [line width=1.2pt, color=VerdeUABC!90] 
            ($(current page.north west) + (0.7cm,-0.7cm)$) rectangle ($(current page.south east) + (-0.7cm,0.7cm)$);
        \draw [line width=0.6pt, color=VerdeUABC!60] 
            ($(current page.north west) + (0.85cm,-0.85cm)$) rectangle ($(current page.south east) + (-0.85cm,0.85cm)$);
    \end{tikzpicture}
}

%==================================================================
% INICIO DEL DOCUMENTO
%==================================================================
\begin{document}

% 1. PORTADA
\begin{titlepage}
    \AddToShipoutPictureBG*{\MarcoPagina}
    \begin{center}
        \includegraphics[width=4cm]{../recursos/escudo-actualizado-2022.png}
        \hspace{1cm}
        \includegraphics[width=5cm]{../recursos/1593484343817.jpg}
        \vspace{1.5cm}

        \LARGE \textbf{Universidad Autónoma de Baja California}\\
        \Large Facultad de Ingeniería Mexicali\\
        \Large Ingeniería en Computación\\
        \vspace{0.7cm}

        {\color{VerdeUABC}\rule{\linewidth}{0.5mm}} \\
        \large \textbf{Reporte de Práctica: \numeroPractica}\\
        \vspace{0.1cm}
        {\Huge {\tituloPractica}}\\
        {\color{VerdeUABC}\rule{\linewidth}{0.5mm}}
        \vfill
        Organización y Arquitectura de Computadoras\\
        \profesor \\
        \vspace{2cm}
        \integranteA \\
        \vspace{1.4cm}
        \textbf{Fecha de entrega:} \\ \fechaEntrega \\
        \vspace{1cm}
        {\color{VerdeUABC}\rule{4cm}{0.1mm}}\\
        Mexicali, Baja California\\ \today
    \end{center}
\end{titlepage}

% ÍNDICE
\newpage
\AddToShipoutPictureBG{\MarcoPagina}
\tableofcontents

% 2. INTRODUCCIÓN
\newpage
\section{Introducción}
Escribe aquí el contexto general de la práctica y un resumen de lo que se abordará.

% 3. PROBLEMÁTICA
\section{Problemática}
Describe el reto técnico o el problema específico que se busca resolver en esta sesión de laboratorio.

% 4. JUSTIFICACIÓN
\section{Justificación}
Explica por qué es importante resolver esta problemática y cómo ayuda a entender la arquitectura de computadoras.

% 5. OBJETIVOS DE LA PRÁCTICA
\section{Objetivos de la práctica}
\begin{itemize}
    \item \textbf{General:} Diseñar e implementar...
    \item \textbf{Específicos:}
    \begin{itemize}
        \item Manipular registros del procesador...
        \item Implementar lógica en ensamblador inline...
    \end{itemize}
\end{itemize}

% 6. MARCO TEÓRICO
\section{Marco teórico}
Conceptos clave como: Pila de registros, FPU, direccionamiento indirecto, etc.

% 7. PROCEDIMIENTO
\section{Procedimiento}
Describe paso a paso lo realizado. Puedes incluir diagramas de flujo o bloques.

La instrucción \inst{MOV} mueve datos del registro \reg{EAX} al registro \reg{EBX}. El valor \hex{0x1A2B} en binario es \bin{0001101000101011}.

\subsection{Ejemplo de Formato de Instrucción}
La instrucción completa sería: \instrformat{MOV}{EAX, 0x1A2B}

\subsection{Tabla de Estado de Registros}
A continuación se muestra el estado de los registros antes y después de la ejecución:

\begin{table}[H]
    \centering
    \caption{Estado de registros antes y después de la ejecución.}
    \label{tab:registros}
    \begin{tabular}{@{}RVRV@{}}
        \toprule
        \multicolumn{2}{c}{\textbf{Antes}} & \multicolumn{2}{c}{\textbf{Después}} \\
        \cmidrule(r){1-2} \cmidrule(l){3-4}
        \textbf{Registro} & \textbf{Valor} & \textbf{Registro} & \textbf{Valor} \\
        \midrule
        EAX & 0x00000000 & EAX & 0x0000001A \\
        EBX & 0x00000000 & EBX & 0x0000002B \\
        ECX & 0x00000000 & ECX & 0x00000000 \\
        EDX & 0x00000000 & EDX & 0x00000000 \\
        \bottomrule
    \end{tabular}
\end{table}

\subsection{Tabla de Formato de Instrucción}
\begin{table}[H]
    \centering
    \caption{Formato de instrucción MOV en x86.}
    \label{tab:formato_instr}
    \begin{tabular}{@{}cccccc@{}}
        \toprule
        \textbf{Bits} & \textbf{31-24} & \textbf{23-16} & \textbf{15-8} & \textbf{7-0} \\
        \midrule
        Opcode & 1000 1001 & \multicolumn{2}{c}{ModR/M} & Immediate \\
        Hex & 89 & \multicolumn{2}{c}{C3} & 00 \\
        \bottomrule
    \end{tabular}
\end{table}

\subsection{Código en Ensamblador}
\begin{lstlisting}[style=assemblyStyle, caption=Ejemplo de código Assembly x86, label=lst:asm_example]
section .data
    msg db 'Hola Mundo', 0xA
    len equ $ - msg

section .text
    global _start

_start:
    ; Escribir mensaje
    mov eax, 4          ; sys_write
    mov ebx, 1          ; stdout
    mov ecx, msg        ; mensaje
    mov edx, len        ; longitud
    int 0x80            ; llamada al sistema
    
    ; Salir
    mov eax, 1          ; sys_exit
    xor ebx, ebx        ; codigo 0
    int 0x80
\end{lstlisting}

\subsection{Código en C}
\begin{lstlisting}[style=cStyle, caption=Integración de ensamblador inline en C, label=lst:c_example]
#include <stdio.h>

int main() {
    int resultado;
    int a = 5, b = 3;
    
    // Ensamblador inline
    __asm {
        mov eax, a
        add eax, b
        mov resultado, eax
    }
    
    printf("Resultado: %d\n", resultado);
    return 0;
}
\end{lstlisting}

\subsection{Código Mixto C con Assembly}
\begin{lstlisting}[style=cStyle, caption=Ejemplo de función con assembly embebido, label=lst:mixed_example]
int suma_asm(int x, int y) {
    int res;
    __asm {
        mov eax, x
        add eax, y
        mov res, eax
    }
    return res;
}
\end{lstlisting}

% 8. RESULTADOS CON PRUEBAS
\section{Resultados con pruebas}
Inserta aquí capturas de pantalla de la ejecución, tablas de datos o fotos del hardware si aplica.

% \subsection{Captura del Debugger}
% \begin{figure}[H]
%     \centering
%     \includegraphics[width=0.9\textwidth]{../recursos/debugger_screenshot.png}
%     \caption{Vista del debugger de Visual Studio mostrando el estado de los registros durante la ejecución.}
%     \label{fig:debugger}
% \end{figure}

% \subsection{Salida del Programa}
% \begin{figure}[H]
%     \centering
%     \includegraphics[width=0.8\textwidth]{../recursos/resultado_ejecucion.png}
%     \caption{Captura de pantalla de la ejecución exitosa del programa.}
%     \label{fig:resultado}
% \end{figure}

\subsection{Configuración de Visual Studio para Capturas}
Para obtener capturas óptimas del debugger en Visual Studio:
\begin{itemize}
    \item Activar vista de registros: \texttt{Debug $\rightarrow$ Windows $\rightarrow$ Registers}
    \item Mostrar desensamblado: \texttt{Debug $\rightarrow$ Windows $\rightarrow$ Disassembly}
    \item Vista de memoria: \texttt{Debug $\rightarrow$ Windows $\rightarrow$ Memory}
    \item Asegurar que los puntos de interrupción estén visibles en la captura
\end{itemize}

% 9. CONCLUSIONES
\section{Conclusiones}
Reflexiones finales sobre el aprendizaje obtenido y el cumplimiento de los objetivos.

% REFERENCIAS
\newpage
\begin{thebibliography}{9}
    \bibitem{ref_irvine} 
    Irvine, K. (2019). \textit{Assembly Language for x86 Processors}. Pearson.
    
    \bibitem{ref_patterson}
    Patterson, D. A., \& Hennessy, J. L. (2017). \textit{Computer Organization and Design}. Morgan Kaufmann.
\end{thebibliography}

% ANEXOS (opcional)
\newpage
\appendix
\section{Repositorio de código}
El código completo de las prácticas y programas utilizados se encuentra disponible en el siguiente repositorio de GitHub:

\begin{center}
\url{https://github.com/usuario/arquitectura-computadoras-practica-XX}
\end{center}

\end{document}